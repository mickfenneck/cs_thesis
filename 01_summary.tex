\chapter*{Summary} % senza numerazione
\label{Summary}



\addcontentsline{toc}{chapter}{Sommario} % da aggiungere comunque all'indice

Nowadays companies are getting more and more connected with potential customers with the help of social networks. These instruments are not only supposed to help people getting linked and chat together but also to give public figures and companies a way to show themselves and to get promoted. Social Networks directly provide services for advertisement: Facebook Ads for Facebook, Twitter Ads for Twitter etc. These services are developed to help people and organizations promoting themselves on that particular platform and usually aren't very useful as a platforms itselves to manage an entire lead generation campaign. Companies are supported only in the brand and products promotion, meaning that the ones that want to find potential customers are generally left alone. The lead generation process, the generation of consumer interest or inquiry into products or services of a business\footnote{\url{https://en.wikipedia.org/wiki/Lead\_generation}}, requires company to have a list of potential customers and, most important, a way to connect with them.
\bigskip


The aim of this project is to show a practical solution to connect real companies with their social pages. Such solution is meant to be implemented in a bigger platform for lead generation that will help companies in the path of finding other businesses that may be interested in buying specific products. The solution shown connects Italian companies with their Facebook page in order to enrich companies' description on Atoka\footnote{\url{https://atoka.io/}}, a leading tool developed by SpazioDati\footnote{\url{http://spaziodati.eu/en/}} that collects structured data and information of all the 6 million companies currently opened in Italy.


  Sommario è un breve riassunto del lavoro svolto dove si descrive l'obiettivo, l'oggetto della tesi, le 
metodologie e le tecniche usate, i dati elaborati e la spiegazione delle conclusioni alle quali siete arrivati.  

Il sommario dell’elaborato consiste al massimo di 3 pagine e deve contenere le seguenti informazioni:
\begin{itemize}
  \item contesto e motivazioni 
  \item breve riassunto del problema affrontato
  \item tecniche utilizzate e/o sviluppate
  \item risultati raggiunti, sottolineando il contributo personale del laureando/a
\end{itemize}




