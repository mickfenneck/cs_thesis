\chapter{Experiment} 
This final chapter is meant to give some explanation of why and how some technical decision were taken.

\section{GraphAPI Requests, Names, DNS}
GraphAPI are the proprietary Facebook REST API that developers need to use if they want to access Facebook's graph. 

When we started to use the APIs we immediately found some problems:

\begin{itemize}
\item Facebook isn't really clear about how many requests a single developer user/app can ask per second, minute, hour or day.
\item Facebook isn't clear about how many request from the same IP can be asked in a single application.
\item Facebook isn't clear at all about how many applications can run with similar requests at the same time
\item Facebook is really foggy in the explanation of what you can ask through the query
\end{itemize}

\subsection*{Requests and time}
As explained in the solution section about GraphAPI, Facebook isn't clear about how many requests a single developer user/app can make
\subsection*{Requests and place}
\subsection*{Requests and applications}
\subsection*{Requests and query content}


perche tanti nomi e non uno singolo
sistema di DNS per non essere bannati

\section{Why blocking isn't useful}
perche blocking e' inutile
